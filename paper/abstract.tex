\begin{abstract}

This paper presents the wide area interactive shell (WISH), a novel wide area
network service and toolkit that gives users the ability to interactively
control large numbers of hosts via a single UNIX shell.  WISH provides a
global process, file, and environment variable namespace in which to spawn,
join, signal, communicate, and synchronize with remote processes. Users
harness these features to carry out complex distributed tasks which are
difficult with normal remote shells.  The main contribution of WISH is that it
empowers users to control multiple hosts by extending familiar shell
programming concepts into the wide area.

We motivate the need for WISH using our experience with managing hosts in
PlanetLab, where some problems we faced could not be easily solved using
remote shells or with existing scalable management solutions.  From this
experience, we describe the desired set of capabilities WISH should expose
to users. We then automate multiple host management
functions using WISH shell scripts in order to demonstrate the new paradigms
WISH offers to users.

\end{abstract}

\vspace{-1ex}
