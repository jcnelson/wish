\Section{Evaluation} \label{sec:eval}

We present a preliminary evaluation of  WISH by implementing two host
management functions as case studies. Using these examples, we demonstrate
many of WISH's unique features.  We argue that these tasks are easier to
perform using a standard UNIX shell in conjunction with WISH than with SSH
alone.

The first case study is implemented in Figure~\ref{fig:ex1}. The script causes
each host within a WISH federation to generate an x509 certificate signing
request with \texttt{openssl}, and then determines which hosts failed to
generate the certificate.  The second case study is implemented in
Figures~\ref{fig:ex2} and~\ref{fig:ex3}.  The script allows a user to alter the running state of
a service on many hosts at once.  Both functions may be implemented by means
other than WISH, but WISH's capabilities make tasks like this easier to
program.


\begin{SaveVerbatim}{ex1}
gpids=$(seq 1 $(nget -n))
keydir="/etc/pks/tls/private/"

for in in $gpids; do
  host=$(nget $i)
	
  pspawn -g $i -i /tmp/csr.dat \ 
	-o /dev/null               \
    -c "openssl req -new -key  \
    $keydir/localhost.key" $host
done

for i in $gpids; do
  if [[ $(pjoin $i) != 0]]; then
    echo "Key generation failed \
    on $(nget $i)"
  fi
done
\end{SaveVerbatim}


\begin{figure}
	\fbox{
		\begin{minipage}{20em}
			\small
			\BUseVerbatim{ex1}
			\caption{OpenSSL CSR Generation}
			\label{fig:ex1}
		\end{minipage}
	}
\end{figure}






\Subsection{Process Management}

The difficulty in running \texttt{openssl} to generate x509 certificate
signing requests \textit{en masse} is that \texttt{openssl} requires user
input to do so.  In our solution, we put the user input into a local file
called \texttt{/tmp/csr.dat}, and spawn a process on each host to invoke
\texttt{openssl}, but redirect \texttt{/tmp/csr.dat} into \texttt{openssl} for
standard input.  We then join with each \texttt{openssl} process and report
any hosts that failed to generate the certificate signing requests.

It is possible to use SSH and \texttt{expect} to issue the correct user inputs
to \texttt{openssl}, but this requires the user to be familiar with
\texttt{expect}.  It is also possible to distribute \texttt{/tmp/csr.dat} to
each host and then redirect the local copy to \texttt{openssl}'s standard
input, but this introduces the requirement that the user (re)distribute
\texttt{/tmp/csr.dat}  if it changes.  If the user ran \texttt{openssl} within
SSH and wanted to know which hosts failed to execute it, the user would either
need to wrap the invocation of \texttt{openssl} with a check on its exit code,
or execute each invocation of \texttt{openssl} serially and check the exit
code of SSH.  With WISH, a user can instead run each invocation of
\texttt{openssl} in parallel and not only capture its exit code, but block
until all tasks have completed or failed.


\begin{SaveVerbatim}{ex2}
while true; do
  sleep 5
  val=$(ggetenv SYS_STATUS)
 
  if [[$val != "done"]]; then
    /etc/init.d/project $val
    ggetenv RC_$(hostname) $?
    psync $(seq 1 $(nget -n))
    gsetenv SYS_STATUS done
  fi
done
\end{SaveVerbatim}


\begin{figure}
	\fbox{
		\begin{minipage}{20em}
			\small
			\BUseVerbatim{ex2}
			\caption{\texttt{mon.sh}: Process Watchdog}
			\label{fig:ex2}
		\end{minipage}
	}
\end{figure}




 


\begin{SaveVerbatim}{ex3}
for i in $(seq 1 $(nget -n)); do
  pspawn -g $i -f mon.sh $(nget $i)
done
\end{SaveVerbatim}


\begin{figure}
	\fbox{
		\begin{minipage}{20em}
			\small
			\BUseVerbatim{ex3}
			\caption{Process Watchdog Launch Script}
			\label{fig:ex3}
		\end{minipage}
	}
\end{figure}






\Subsection{Global Environment Variables}

Most Linux services are controlled via initscripts, which are used to start,
stop, and restart them~\cite{linuxbase}.  The difficulty of running many
initscripts for the same service at once is verifying that each script
performed as expected.  Our WISH solution in Figure~\ref{fig:ex3} deploys the
long-running shell process called \texttt{mon.sh}, detailed in
Figure~\ref{fig:ex2}, to address this problem.  \texttt{mon.sh} periodically
checks the global environment variable \texttt{SYS\_STATUS} and invokes the
operation stored as its value on the initscript \texttt{/etc/init.d/project}.
It records the exit code of the initscript in a global environment variable
prefixed with \texttt{RC\_} and named after its hostname, waits for every
other \texttt{mon.sh} instance to do the same, and then resets
\texttt{SYS\_STATUS} to ``\texttt{done}".

It is possible to use SSH, Plush, or pssh to invoke an initscript across a set
of nodes, but more difficult.  The user must somehow make the initscript
invocation report back its success or failure, and additionally have a means
of knowing when all invocations have completed.  WISH efficiently solves both
problems by allowing the invocation to share status information with the user
via global environment variables.
