\Section{Motivation} \label{sec:motivation}

More often than not, distributed systems deployed on PlanetLab run in a
\textit{slice}, a collection of lightweight virtual machines spread across an
equal number of nodes~\cite{PlanetLab-architecture}.  Within a slice, a single
VM, or \textit{sliver}, is subject to naturally occurring  host faults
(including complete VM re-instantiation), network faults, and resource
scarcities, all of which can lead to unexpected system behavior.  To counter
these faults, developers devise means to monitor the health of their systems
and to perform diagnostic and administrative tasks on their misbehaving
slivers. These tasks often include reinstalling, reconfiguring, and upgrading
their software~\cite{PlanetLab-website}.  While there are existing systems
such as CoMon~\cite{CoMon}, SWORD~\cite{SWORD}, Stork~\cite{Stork}, and
Puppet~\cite{Puppet} that can handle these tasks, they each present an
additional deployment challenge to a PlanetLab user who is unfamiliar with
them and who only needs a small subset of their functionality.

In the process of designing WISH, we considered what expectations we could
reasonably have of the typical PlanetLab user. 
By their very nature, PlanetLab users should be familiar with remote shells and
with UNIX shell programming.  Since they use PlanetLab to deploy and test
experimental distributed software, they can also be expected to have varying
degrees of familiarity with distributed programming paradigms and
design patterns.  Since the default means of contacting and
administrating PlanetLab slivers is via SSH~\cite{PlanetLab-architecture}, we
are motivated to create a tool that lets a user extend familiar shell concepts
such as job management, I/O redirection, and environment variables to operate
on a collection of hosts instead of a single host.

The challenges PlanetLab users face in managing their slices are not specific
to PlanetLab's architecture.  Administering collections of hosts in both
clouds and wide-area networks requires the ability to monitor the hosts'
states and the ability to issue commands to hosts efficiently.  While there are
many tools that provide one or both abilities, none to our knowledge leverage
both a user's shell programming and distributed computing experience as the primary
means of performing administrative functions.
