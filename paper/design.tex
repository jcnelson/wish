\Section{Design} \label{sec:design}

We designed WISH to not hide the fact that there are multiple hosts within the
WISH federation.  There are existing systems such as Plush~\cite{plush} and
pssh~\cite{parallel-ssh} that provide an interface for interactively issuing
shell commands to many hosts, but they are limited in their expressiveness
because they issue commands synchronously, in lock-step.  This approach
simplifies shell programming by treating a collection of hosts as a single
host, but it requires the user to issue the hosts the same commands.  This is
not always desirable.

A more flexible approach is to treat each host as a receptacle in which a user
can place processes and data, so users can orchestrate host behavior however
they desire.  This can be partially achieved by iterating over a list of hosts
and issuing the hosts commands in parallel via a remote shell.  However, this
solution offers no easy way for the processes to share data and offers no easy
way for the user to control them once they start.

Given this approach, we designed WISH to provide users with the following
capabilities.

\begin{description}

\item[Process Control:] A user can spawn, signal, join, and synchronize with
processes on any host within their WISH instance.

\item[I/O Redirection:] A user can redirect process input and output between
any hosts in a WISH instance.

\item[Resource Monitoring:] A user can monitor the resource and network
utilizations of hosts.

\item[Environment Variables:] A user can get, set, and atomically test-and-set
globally-visible shell environment variables.

\item[File Access:] A user can globally expose files to all other hosts within
the same WISH instance. Remote hosts can download files which are visible to
it.

\end{description}

Below, we describe how WISH meets these requirements.

\Subsection{Overview}

WISH uses a federated architecture for communication between its member hosts.
There is no federation leader, but each member of the federation implements a
\textit{global shell environment} to coordinate execution and communication
between the jobs it has been instructed to manage.  The global shell
environment exposes environment variables, processes, and files that are
visible to jobs originating from the same daemon, wherever they are executing.
In practice, each host runs a WISH daemon to participate in the federation,
and its daemon assumes responsibility for processes spawned by local users.  

To create a federation, a user distributes the WISH daemons and client
programs to his/her hosts out-of-band, and generates a list of hostnames and
port numbers for each daemon to use to contact the other daemons.  In future
work, we intend to allow hosts to authenticate with and securely join existing
federations and to learn of other joining daemons via a distributed
gossip protocol.

\Subsection{Client and Commands}

Users do not directly interact with the WISH daemon. Instead, WISH provides
shell commands and a shell client which communicate with the local WISH daemon
via the loopback network interface.  The WISH shell client is a program that
runs as a child of a UNIX shell process that receives and prints output from
WISH processes to the user's TTY.  We plan to make the client configurable
such that data from different remote hosts may be color-coded (if the TTY
supports it), prefixed by the hostname or a host alias, and selectively
filtered or redirected as the user sees fit.

\Subsection{Peers and Heartbeats}

The WISH daemon attempts to monitor the health of all other hosts within the
federation. When a host joins the WISH federation, its daemon attempts to
connect to as many other daemons as possible in order to send and receive
``heartbeat" packets. The heartbeat packets contain information about a host's
resource utilization.  Once it connects to its peers within the federation, it
periodically re-sends heartbeat packets to them, and keeps a bounded log of
the last heartbeats it has received from each other host connected to it.  The
heartbeat information may be queried by the user to select hosts with
desirable resource availabilities.  The information maintained by each daemon
includes a remote host's average CPU load in the past 5, 10, and 15 minutes,
the amount of free RAM and disk space, and the average round-trip times of the
previously-received heartbeat packets from each other host.

WISH provides a command called \texttt{nget} to determine the host with the
$i$th most free RAM, the $i$th most free disk space, the $i$th lowest CPU load
in the past 5, 10, and 15 minutes, or  the $i$th lowest network latency (for
all 0 < $i$ < $N$, the number of hosts).  This is useful for when the user
does not care which specific hosts are used to process a job, but only needs a
subset with available capacity.  Additionally, the \texttt{nget} program lets
a user query the total number of hosts in the federation, so the user can iterate
through the hosts in the order specified.

\Subsection{Remote Files}

Since WISH jobs run across multiple hosts, a user may need to distribute data
between them.  To facilitate this, WISH provides an embedded HTTP(S) server to
serve files from a predetermined document root.  As a security measure, it
maintains a blacklist of directories within the document root from which no
files may be served.  Users can blacklist and un-blacklist local files and
directories with the \texttt{fhide} and \texttt{fshow} commands, respectively.

\Subsection{Process Management}

When a user wishes to create a process to run a job, the local WISH daemon
starts a \textit{global process} by registering the job with its global shell
environment and assigning it a globally-unique~\footnote{In practice, a GPID
is either a user-selected or randomly-generated 64-bit value, making it unique
with high probability.} GPID (global process identifier).  It then contacts
the user-indicated host in the federation on which to run the job and forwards
the job's commands to that remote daemon.  The remote daemon performs some
bookkeeping, forks a child process to run the job, and acknowledges the local
daemon's job execution request.  When the child process terminates, the remote
daemon forwards the exit code and last signal received of the child process
back to the local daemon, completing the global process's life.  We refer to
the set of actions the local daemon takes to manage a process as
\textit{spawning} the process, and the set of actions taken by the remote
daemon as \textit{executing} the process.

The remote child process must know how to contact the local daemon to
participate in its shell environment.  Part of the bookkeeping the remote
daemon performs before forking the child includes setting a few variables in
the child process's environment which tell the child the hostname and port
number of the daemon that spawned it.  This information will be preserved
across future spawns performed by this child, so every descendent process of
the child will access the global shell environment on the local daemon that
spawned it (unless explicitly overridden by the user).

There are four commands to manage jobs in WISH.  The \texttt{pspawn} command
instructs the local daemon to spawn a process and execute it on a remote
daemon of the user's choosing, and may either specify a list of shell commands
to evaluate or a locally-hosted binary file to fetch, download, and then
execute.  The \texttt{psig} command routes signals from the local daemon to
the remote daemon executing a process, which in turn signals the process
itself.  The \texttt{psync} command requests or acknowledges process barriers,
and will not exit until all other processes named in the list of GPIDs passed
to it perform the same barrier request.  Finally, the \texttt{pjoin} command
joins with an existing process by waiting until the local daemon has the
process's exit code, and then retrieving and printing it out.

\Subsection{IO Redirection}

When a WISH daemon executes a process, it directs its standard output and
standard error files to local, temporary files.  As the process executes the
user's job, the daemon reads and sends back the data stored in the redirected
output and error files to the daemon that spawned the process as quickly as it
can.  Once the data are received by the daemon that spawned the process, they
are forwarded to the user's client, which prints them out as the user desires.

Sometimes, a user may need to redirect a remotely executing process's standard
out and/or standard error.  To do this, the user identifies a file for each
stream in the \texttt{pspawn} command, which causes the daemon to redirect the
process output accordingly.  Also, a user may need to specify a file to serve
as standard input for the remote process.  To do this, the user identifies a
file to \texttt{pspawn} to be used as standard input.  If this file is visible
in the local daemon's global shell environment, the remote daemon downloads
the file to temporary storage and opens it as the process's standard input.
It removes the file once the process terminates.

\Subsection{Environment Variables}

A process running in WISH has access both to its local environment variables
and the environment variables provided by its originating daemon's global
shell environment.  A user can get, set, and atomically test-and-set these
global environment variables in the global shell environment via the
\texttt{ggetenv}, \texttt{gsetenv}, and \texttt{taset} commands, respectively.
These commands search their local environment for a WISH daemon's hostname and
port number, and if present, contact the daemon to carry out the intended
operation (if not present, they contact localhost at the port number in the
local WISH configuration).  This way, all descendents of a process started by
the user's local daemon manipulate the same environment variable space as the
user.

