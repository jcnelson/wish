\appendix \section{Appendix: WISH Commands} \label{sec:appendix}

In this section, we provide a brief listing of commands supported by the WISH
daemon.

\begin{description}

\item[dctl [start|stop|restart|reset{]}]\hfill \\
 Start/restart/reset/stop a node's shell daemon

\item[fhide [-h HOST[:PORT{]}{]} PATH [PATH...{]}]\hfill \\
 Make a file or directory tree that is visible to the system invisible again.
HOST and PORT refer to the WISH daemon which spawned the job, if it was not
spawned on localhost.

\item[fshow [-h HOST[:PORT{]}{]} PATH [PATH...{]}]\hfill \\
Make a file or directory tree visible (but read-only) to
the system (globally visible to all nodes, but identified on other nodes by the
path and by the handle of the node on which it resides)

\item[getgpid [-h HOST[:PORT{]}{]} PID] \hfill \\
Retrieve the global PID when given a local PID

\item[ggetenv [VAR{]}]\hfill \\
Get a global environment variable from the system (the
variable could have been set remotely by another process)

\item[gsetenv [VAR VALUE{]}]\hfill \\
Set a global environment variable in the system
(globally visible to all nodes)

\item[nget [-r|-l|-d|-c|-n{]} [-h| HOST[:PORT{]}{]} RANK]\hfill \\
Return the NID of a node with desired characteristics of rank RANK that can be used to
identify the node for
the duration of the process's life (for example RANK 0 is always the
fastest/best)\\  
	-l            lowest latency \\
   -r             highest free RAM \\
   -d             highest free disk \\
   -c             highest free CPU \\
   -h HOST[:PORT] Access the daemon running on HOST[:PORT] \\
   -n             Don't print a host; print the number of nodes. \\
                  If this option is given, RANK is ignored \\
   RANK           The rank the desired host must have (0 being the highest/best) \\

%\item[nrel [NID{]}]\hfill \\
%Release a node (will be implicitly done when the process
%dies), so it can be chosen again by running nget within the job.

\item[pjoin [-n{]} [-h HOST[:PORT{]}{]} GPID]\hfill \\
Join with a process (either local or remote, using
the handle). This will receive the process's stdout and stderr contents, as
well as return code.

\item[psig [-h HOST[:PORT{]}{]} -SIG GPID]\hfill \\
Send signal SIG to a process (either local or
remote, using the handle)

\item[psigall [-h HOST[:PORT{]}{]} -SIG]\hfill \\
Send SIG to all GPIDs

\item[pspawn [-d{]} [-t TIMEOUT{]} [-h HOST[:PORT{]}{]} [-g GPID{]} [-f FILE{]} [COMMAND{]}]\hfill \\
Spawn a process to run a job (which is
either a list of commands or an executable file), but get back a
globally-unique PID (a GUPID) to it so the user can manipulate it (i.e. signal
it). Optionally spawn the process on a remote host (user-chosen or
system-chosen). (NOTE: | means "or" in this documentation).

\item[psync [-h HOST[:PORT{]}{]} [-t TIMEOUT{]} GPID [GPID...{]}]\hfill \\
Tell a list of processes to synchronize
execution via a barrier (processes can be local or remote)

\item[rin [NID:STREAM COMMAND|GUPID{]}] 
Redirect a stream of bytes
(or copy a file) to a process's stdin (akin to '<', but works on any
stream/file in the system). STREAM can be commands to be evaluated that produce
output to be redirected. (note: in the documentation, | means 'or'). If a GUPID
is given, then the data is directed to the process's stdin

%\item[rout [NID:STREAM|NID:PATH{]} [COMMAND|GUPID{]}]\hfill
%Redirect a stream (or copy
%a file) to a file on a particular node or a particular stream of bytes (which
%can be a series of commands that can receive input from the command or
%process). If a GUPID is given, then the data is redirected from the
%corresponding process.

\item[taset [NAME VALUE CMP{]}]\hfill \\
Atomically test and set a global environment
variable (perform test COMP on VAR with VALUE, where COMP is an appropriate
shell comparison like -eq). VAR can be a global environment variable.

\end{description}



