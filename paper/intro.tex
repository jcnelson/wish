\Section{Introduction} \label{sec:intro}

In this paper we present our design of the wide area interactive shell (WISH),
a distributed computing service which provides the ability to scalably execute
shell commands across wide area networks. WISH is comprised of  a daemon and a
set of command-line tools which allow any standard UNIX shell to spawn and
manage \emph{jobs}\footnote{We use the term ``job" to refer to a sequence of
one or more commands that a user submits to a conventional UNIX shell.  A job
executes in one process initially, but may fork others.} across a large set of
hosts.

WISH is targeted at individuals who need more powerful job controls than those
which are available in  traditional remote shells such as SSH.  For example,
with WISH a user can quickly set up a publish-subscribe system by spawning a
process on each host.  This process will periodically query a global
environment variable and execute commands based upon its value.  As another
example, a user can quickly distribute large datasets to hosts by scripting a
subset of hosts to download the dataset from its origin, and then command
disjoint subsets of hosts to download the dataset from peers that already have
it.  As a third example, a user can immediately identify a subset of hosts
with WISH that are available to execute a computationally-intensive task, and
then use their UNIX shell of choice to issue the task to them.

While existing remote shells such as SSH~\cite{OpenSSH} may be leveraged to
execute the above examples, the shell user must implement the inter-job and
inter-host communication infrastructure necessary to ensure correct execution.
WISH's contribution is that it provides this necessary infrastructure by
creating a network-wide shell environment and by exposing synchronization
primitives for a user's job to leverage.  These allow a user to quickly write
and run jobs in a familiar environment across many hosts, and empower users to
leverage their knowledge of distributed programming techniques.

In the remainder of this paper, we describe the motivation for designing WISH
based upon our experience with PlanetLab~\cite{PlanetLab-architecture} slice
administration.  We then present our design of the WISH system and compare it
to existing parallel execution environments.  We conclude with case studies
and a discussion of related and future work.

